\documentclass[12pt,a4paper, norsk]{article}

\usepackage[norsk]{babel}
\usepackage{hyperref}
%opening
\title{Forberedelse prøveeksamen MPS 140, Del 2}
\author{Njål Foldnes}
\date{\today}

\begin{document}

\maketitle


Vi ser på  \href{https://onlinelibrary.wiley.com/doi/10.1111/sjop.12793}{denne studien} som omhandler måling av  psykologiske konstrukter relevante for arbeidsmiljø, slik som autonomi, personlig utvikling, osv. Hele instrumentet kalles People Performance Scales (PPS).  Du kan laste ned datafila på canvas, i filformatet .rds, og i filformat .cvs. Der finner du også variabelforklaringene i en egen .txt fil, samt en pdf av artikkelen.

Finn først ut hvor mange respondenter som har missing data, og fjern disse respondentene fra datasettet.
(Tips: bruk R-funksjonen \textsf{complete.cases())}. 
	
	Da har vi et nytt og redusert datasett. Basert på dette, beregn beskrivende statistikk for 
	\begin{itemize}
		\item alder
		\item kjønn
		\item gjennomsnittlig engasjement (ligger i variabelen {\it Engasjement}) på de seks engasjement-item.
	\end{itemize}

Undersøk også om kjønn påvirker  {\it Engasjement}. 


Undersøk hvordan mellom  y={\it Engasjement} påvirkes av variabelen x={\it Autonomi}, som er gjennomsnittet av de tre autonomi variablene. Beregn også korrelasjonen. Sjekk de tre antagelsene for regresjonen. 

Sjekk hvorvidt regresjonen mellom y={\it Engasjement} og variabelen x={\it Utbrenthet} forandrer seg når vi kontrollerer for kjønn og alder. 

Nå skal vi se på itemene som indikatorer for latente konstrukt. 
Du kan feks lage et nytt datasett som bare inneholder søyle 21 og utover. Lag et korrelasjonsplott for dette nye datasettet med \textsf{cor.plot()} i psych-pakken. 

For engasjement-konstruktet, bestem de 6 standardiserte faktor ladningene.  Hva er reliabiliteten?

Hva blir korrelasjonen mellom engasjement og autonomi når vi behandler dem som latente variable. 
Sammenlikn med korrelasjonen vi fikk når vi brukte gjennomsnittskårene.


I variabelforklaringene (.txt fila) så er utbrenthet målt med 7 indikatorer: Overb1-Overb4 og Kyn1-Kyn3.
Undersøk ved hjelp av \textsf{fa()} funksjonen, der du spesifiserer 2 faktorer, om disse itemene kan lade på to sub-konstrukter (overbelastning og kynisme). Hva er korrelasjonen mellom disse to latente faktorene (overbelastning og kynisme)?
Finn ut hvilken av modellene omtalt på side 114 som framkommer hvis vi skiller overbelastning og kynisme.

Forskerne ender opp med å anbefale en 15-faktor struktur for PPS skalaen. Kjør en \textsf{fa()} analyse på alle 57 indikatorene der du spesifiserer 15 faktorer. 
Vurder hvorvidt strukturen matcher 15 distinkte faktorer som hver har sine naturlige indikatorer. 


Kjør en parallel-analyse på alle indikatorene og vurder hvor mange konstrukter PPS skalaen består av.


\end{document}
