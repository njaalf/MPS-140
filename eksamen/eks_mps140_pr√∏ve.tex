  \documentclass[12pt,a4paper, norsk]{article}

 \usepackage[norsk]{babel}
 \usepackage{color}
	 \usepackage{multicol, multirow}
  \usepackage[utf8]{inputenc} 
\usepackage{pgfplots} 
\usepackage{hyperref}
 \usepackage{url}
 \usepackage{enumerate}
 \usepackage{graphicx} 
 \usepackage{tabularx}
 \usepackage{wrapfig} 
 \usepackage{subfigure}
 \usepackage{amsmath,amssymb}
\usepackage{multirow}
\usepackage{tikz}
\usepackage{times}

\addtolength{\textwidth}{0.5in}

\title{Prøveeksamen MPS - redusert versjon}
\date{\nodate}
 
 \begin{document} %\color{blue}
 	
 	Eksamen er todelt. Første del inneholder oppgaver som skal besvares basert på informasjon i dette oppgavesettet. 
 	Andre del inneholder spørsmål omkring analyser av datasettet som ble levert ut to uker før eksamen. 
 	Denne prøveeksamenen inneholder omkring halvparten så mange oppgaver som den kommende eksamen. 
 	Oppgavene i denne prøveeksamen er representative med tanke på tematikk og vanskegrad. 
 	
 \section*{Del 1}
 
Personlighetstester blir ofte brukt i rekruttering i arbeidslivet. 
Formålet er å finne ut hvilke personlighetstrekk en jobbkandidat har.  En stor bedrift lyste ut ti ledige stillinger som kundekonsulent. Totalt var det $n=295$ søkere til stillingene, og alle tok en personlighetstest. 
Testen resulterte blant annet i scorer på følgende personlighetstrekk:
\begin{itemize}
	\item {\it Medmenneskelighet}: Et mål på hvor stor tillit man har til medmennesker, og hvor stor omtanke man har for medmennesker.  Måles på en kontinuerlig skala fra 0-100.
	
	\item {\it Nevrotisisme}:  Et mål på følelsesmessig ustabilitet med tilbøyelighet til angst, irritasjon og depresjon. Måles på en kontinuerlig skala fra 0-100.
	
\end{itemize}

I tillegg skal vi se på følgende variabler: {\it Kjønn}, {\it Alder} og {\it Utdanningsnivå}.  Sistnevnte variabel har følgende fem kategorier: Ungdomsskole, Videregående skole, Årsstudium innen høyere utdanning, Bachelorgrad og Mastergrad. 

 


\subsection*{Oppgave 1  (4 poeng) }



 \begin{enumerate}[(a)]


\item På hvilket målenivå er variabelen {\it Utdanningsnivå}?


\item Krysstabellen mellom {\it Kjønn} og {\it Utdanningsnivå} er som følger:



\begin{table}[ht]
	\centering
	\begin{tabular}{rrrrrr}
 & Ungdomsskole & Vid.g. skole & Årsstudium & Bachelor & Master \\ 
  \hline
Mann &  13 &  11 &  43 &  ? &  18 \\ 
  Kvinne &  11 &  22 & 102 &  35 &  29 \\ 
   \hline
\end{tabular}
\end{table}

Hvor mange menn hadde Bachelor som sitt utdanningsnivå?

\item 
Hvor stor prosentandel av søkerne hadde Mastergrad?



\item Lag en graf som viser fordelingen til {\it Utdanningsnivå} for de kvinnelige søkerne.



	\end{enumerate}		

\subsection*{Oppgave 2  (4 poeng) }
Figur \ref{fig:1}  viser fordelingen til {\it Medmenneskelighet} hos menn og kvinner.
\begin{figure}[htp]
	\centering
	\includegraphics[width=0.99\textwidth]{fig1}
	\caption{Oppgave 2 }
	\label{fig:1}
\end{figure}

 \begin{enumerate}[(a)]

\item Hva kalles den type graf som vises i figuren? 


\item  Hvilket kjønn  scorer høyest på medmenneskelighet?

\item  Anslå medianverdien til medmenneskelighet hos kvinner.


\item  Anslå interkvartilbredden (IK) i medmenneskelighet hos menn. 





\end{enumerate}		


\subsection*{Oppgave 3  (2 poeng) }
La oss anta at  {\it Nevrotisisme} hos en tilfeldig valgt 
søker er normalfordelt med  $\mu=50$   og standardavvik $\sigma=15$. 



\begin{enumerate}[(a)]
	
	\item  Lars har en nevrotisisme-score på 80. Hva er $z$-verdien til denne scoren?
	
	\item Hva er sannsynligheten for at en tilfeldig valgt søker scorer mellom 20 og 80 i nevrotisisme? Skisser sannsynligheten som arealet under en graf.
	

	\end{enumerate}		
%




	
	
\subsection*{Oppgave 4  (3 poeng) }
	
	Tabellen under  inneholder nøkkeltall for score på medmenneskelighet for menn og kvinner. 
	\begin{figure}[htp]
		\centering
		\includegraphics[width=0.9\textwidth]{fig2}
		\caption{Oppgave 5.}
		\label{fig:2}
	\end{figure}


	
	 \begin{enumerate}[(a)]
	
	\item  Er tallene i tabellen observatorer eller parametre?
	
	\item Det påstås at kvinnelige søkere  i gjennomsnitt er mer medmenneskelige enn mannlige søkere. Skriv opp hypoteser $H_0$ og $H_A$ for å teste denne påstanden. Bruk symbolene $\mu_K$ og $\mu_M$. 
	
	
	\item Når testen utføres så blir p-verdien $< .001$. Hva blir konklusjonen på testen? Formuler deg i et lettfattelig språk. 
	 
\end{enumerate}		
		
		
\subsection*{Oppgave 5  (5 poeng) }
			Vi ser på regresjonen mellom den uavhengige variabelen $x=${\it Alder} og den avhengige variabelen $y$=
			{\it Medmenneskelighet}.
		 Regresjonstabellen for linja $$\widehat{\text{Medmenneskelighet}}=b_0+b_1\cdot \text{alder}$$ er
			
			\begin{center}
		
	\begin{tabular}{rrrrr}
  \hline
 & Estimate & Std. Error & t value & Pr($>$$|$t$|$) \\ 
  \hline
(Intercept) & 62.06 & 3.38 & 18.36 & 0.00 \\ 
  alder & 0.24 & 0.11 & 2.19 & 0.03 \\ 
   \hline
\end{tabular}
			 	\end{center}
			
			
			 \begin{enumerate}[(a)]
			
			\item  Lise er en ny søker, og hun er 50 år gammel. Hva er prognosen for hennes medmenneskelighets-score? 
			
			\item  Gi en tolkning av koeffisienten $b_1=0.24$.
			
			\item 
			 Det påstås  at søkere har høyere score på medmenneskelighet jo eldre de er. Skriv opp $H_0$ og $H_A$ for å teste denne påstanden. Bruk symbolet $\beta_1$. 
			 
			 \item Har vi tilstrekkelig støtte til å hevde at scoren på medmenneskelighet øker med alder? Bruk signifikansnivå $\alpha=0.05$. 
			
			\item Nevn tre forutsetninger for at vi kan stole på resultatet fra en lineær regresjonsmodell.
			
			
			
				\end{enumerate}		
				
	
	\subsection*{Oppgave 6  (4 poeng) }
	
	Fram til nå har vi antatt at nevrotisisme er målt på  en skala fra 0-100. I forskning brukes skalaer for å måle nevrotisisme, blant annet som del av big 5. Forskerne Weinstock og Whisman lanserte i 2006 en skala for nevrotisisme som består av 8 item, som måles på 4-punkts  Likert skala.
	Itemene kalles (tense, nervous, temperamental, irritable, envious, unstable, insecure, emotional). 
	
	 
	
	
	\begin{enumerate}[(a)]
	
	\item Forklar hvorfor forskerne bruker flere enn ett item. 
	
	\item Reliabiliteten til skalaen var $\alpha=.86$ i et representativt utvalg. Definer reliabilitet og gi en tolkning av $\alpha$ verdien. 
	
	\item Forklar hva validitet er. 
	
	\item Forskerne oppgir at deres skala er korrelert med $r=0.7$ med nevrotisisme skalaen som ofte brukes i big five instrumentet (NEO Personality Inventory ,Costa og McCrae, 1992). 
	Forklar hvordan dette styrker validiteten. Hva kaller vi den type validitet som her styrkes?
	
	
	
	
\end{enumerate}


\subsection*{Oppgave 7  (3 poeng) }

\begin{enumerate}[(a)]
	
	\item Forklar hva en latent variabel er for noe. Bruke gjerne nevrotisisme som eksempel. 
	
	\item Hva er formålet med EFA (eksplorerende faktor analyse). Bruke gjerne nevrotisisme som eksempel.
	
	\item Gi et eksempel på at en skala kan bestå av to subskalaer. Bruke gjerne nevrotisisme som eksempel.
	

	
	
	
\end{enumerate}



\section{Del 2} 







 
  \end{document}